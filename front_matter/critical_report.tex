\documentclass[parskip=full]{scrreprt}

\RedeclareSectionCommand[pagestyle=plain,afterskip=10pt plus 1pt]{chapter}
\setkomafont{chapter}{\mdseries\headingfont\fontsize{40}{40}\selectfont\color{black!80}}
\setkomafont{pageheadfoot}{\normalsize}

\def\pnumbox#1{#1\hspace*{8cm}}
\def\gobble#1{}
\DeclareTOCStyleEntry[
  indent=0pt,
  beforeskip=0pt,
  entryformat=\itshape,
  entrynumberformat=\textcolor{oldred},
  numwidth=2em,
  linefill=\hfill,
  pagenumberbox=\pnumbox,
  pagenumberformat=\itshape
]{tocline}{part}

% use commented values if there are no parts
\DeclareTOCStyleEntry[
  indent=0pt,
  beforeskip=0pt,
  entrynumberformat=\textcolor{oldred},
  numwidth=2em,
  linefill=\hfill,
  pagenumberbox=\pnumbox
]{tocline}{section}

\DeclareTOCStyleEntry[
  indent=0pt,
  beforeskip=-\parskip,
  entrynumberformat=\gobble,
  entryformat=\ltseries,
  numwidth=2em,
  linefill=\hfill,
  pagenumberbox=\pnumbox,
  pagenumberformat=\ltseries
]{tocline}{subsection}

\usepackage{polyglossia}
\setdefaultlanguage{english}

\usepackage{fontspec}

\setmainfont{Source Sans Pro}[
  ItalicFont = Source Sans Pro Italic,
  BoldFont = Source Sans Pro Bold,
  BoldItalicFont = Source Sans Pro Bold Italic,
  FontFace = {lt}{n}{Source Sans Pro Light},
  FontFace = {lt}{it}{Source Sans Pro Light Italic},
  FontFace = {sb}{n}{Source Sans Pro Semibold},
  FontFace = {sb}{it}{Source Sans Pro Semibold Italic},
  Numbers = Proportional,
  Ligatures = Common
]
\DeclareRobustCommand{\ltseries}{\fontseries{lt}\selectfont}
\DeclareRobustCommand{\sbseries}{\fontseries{sb}\selectfont}
\DeclareTextFontCommand{\textlt}{\ltseries}
\DeclareTextFontCommand{\textsb}{\sbseries}

\newfontfamily\headingfont{Fredericka the Great}

\usepackage[pass]{geometry}
\newgeometry{twoside,inner=20mm,outer=40mm,top=20mm,bottom=40mm}

\usepackage{url}
\urlstyle{same}

\usepackage{microtype}
\microtypesetup{verbose=silent}

\usepackage{booktabs,array,longtable}
\newcolumntype{L}[1]{>{\raggedright\let\newline\\\arraybackslash\hspace{0pt}}p{#1}}

\usepackage{graphicx}

\usepackage{xcolor}
\definecolor{oldred}{rgb}{.8313,0,0}

\usepackage{pdfpages}

\usepackage{scrlayer-scrpage}
\pagestyle{scrheadings}
\clearscrheadfoot
\cfoot[\thepage]{\thepage}
\pagenumbering{roman}

\usepackage{enumitem}
\setlist[description]{%
	labelindent=2em,%
	labelwidth=6.5em,%
	leftmargin=8.5em,%
	labelsep=0pt,
	first=\ltseries,%
	font=\normalfont\itshape\ltseries%
}


\makeatletter

\def\@firstofthree#1#2#3{#1}
\def\@secondofthree#1#2#3{#2}
\def\@thirdofthree#1#2#3{#3}
\def\@firstoffour#1#2#3#4{#1}
\def\@secondoffour#1#2#3#4{#2}
\def\@thirdoffour#1#2#3#4{#3}
\def\@fourthoffour#1#2#3#4{#4}
\def\Dotfill{\leavevmode\cleaders\hb@xt@ .75em{\hss .\hss }\hfill \kern \z@}

\def\lyrefnumber#1{\expandafter\@setref\csname r@#1\endcsname\@firstofthree{#1}}
\def\lyreftitle#1{\expandafter\@setref\csname r@#1\endcsname\@secondofthree{#1}}
\def\lyrefpage#1{\expandafter\@setref\csname r@#1\endcsname\@thirdofthree{#1}}

\def\lyrefgenrenumber#1{\expandafter\@setref\csname r@#1\endcsname\@firstoffour{#1}}
\def\lyrefgenregenre#1{\expandafter\@setref\csname r@#1\endcsname\@secondoffour{#1}}
\def\lyrefgenretitle#1{\expandafter\@setref\csname r@#1\endcsname\@thirdoffour{#1}}
\def\lyrefgenrepage#1{\expandafter\@setref\csname r@#1\endcsname\@fourthoffour{#1}}

\def\lyref#1{%
  \begingroup%
  \makebox[0pt][l]{\color{oldred}\lyrefnumber{#1}}\hspace*{2em}%
  \lyreftitle{#1}\Dotfill\lyrefpage{#1}%
  \endgroup%
}
\def\lyrefpart#1{%
	\begingroup%
	\makebox[0pt][l]{\sbseries\color{oldred}\lyrefnumber{#1}}\hspace*{2em}%
	\makebox[0pt][l]{\sbseries\lyreftitle{#1}}\hspace*{6.5em}%
	\hfill\sbseries\lyrefpage{#1}%
	\endgroup%
}
\def\lyrefsection#1{%
	\begingroup%
	\makebox[0pt][l]{\color{oldred}\lyrefgenrenumber{#1}}\hspace*{2em}%
	\makebox[0pt][l]{\ltseries\lyrefgenregenre{#1}}\hspace*{6.5em}%
	\lyrefgenretitle{#1}\Dotfill\lyrefgenrepage{#1}%
	\endgroup%
}
\InputIfFileExists{../out/lilypond.ref}{}{\InputIfFileExists{../lilypond.ref}{}{}}


\newcommand\fancytitlehead{
	\headingfont%
	\fontsize{80}{80}\selectfont\textcolor{black!80}{\@ifundefined{@shortname}{\@lastname}{\@shortname}.}\\[15pt]%
	\fontsize{60}{60}\selectfont\@ifundefined{@shorttitle}{\@title}{\@shorttitle}.%
}


\def\firstname#1{\def\@firstname{#1}}
\def\lastname#1{\def\@lastname{#1}}
\def\shortname#1{\def\@shortname{#1}}
\def\shorttitle#1{\def\@shorttitle{#1}}
\def\namesuffix#1{\def\@namesuffix{#1}}
\def\instrumentation#1{\def\@instrumentation{#1}}
\def\parts#1{\def\@parts{#1}}

\firstname{\relax}
\lastname{\relax}
\namesuffix{\relax}
\instrumentation{\relax}
\parts{\relax}


\def\maketitle{%
\begin{titlepage}%
	\Large%
	{\@titlehead}%
	\vfill%
	{\strut\@firstname}\\%
	{\sbseries\color{oldred}\strut\@lastname}\\%
	{\strut\@namesuffix}%
	\vfill%
	{\sbseries\@title}\\%
	{\@subtitle}\\[\baselineskip]%
	{\itshape\@instrumentation}%
	\vfill%
	{\itshape\@parts}\hspace*{\fill}\raisebox{0pt}[0pt][0pt]{\includegraphics{ees_logo}}%
\end{titlepage}%
}


\newif\iftemplate\templatetrue
\newif\ifprintreport\printreportfalse
\directlua{
scores = {
  ob1 = "Oboe I",
  ob2 = "Oboe II",
  ottoni = "Clarino I, II in C\string\\newline Timpani in C–G",
  vl1 = "Violino I",
  vl2 = "Violino II",
  vla = "Viola",
  coro = "Coro",
  org = "Organo",
  b = "Bassi",
  full_score = "Full Score"
}

last_arg = arg[\string#arg]
texio.write("Last argument: " .. last_arg)
if not (scores[last_arg] == nil) then
  tex.print("\string\\def\string\\lypdfname{" .. last_arg .. "}")
  tex.print("\string\\parts{" .. scores[last_arg] .. "}")
  if (last_arg == "full_score") then
    tex.print("\string\\printreporttrue")
  end
end
}

\@ifundefined{lypdfname}{%
  \templatefalse
  \printreporttrue
  \parts{Draft}
}{\templatetrue}

\makeatother






\begin{document}
\frenchspacing

\titlehead{\fancytitlehead}
\firstname{Ernst Wilhelm}
\lastname{Wolf}
\title{Ostercantate}
\subtitle{Des Lebens Fürſten haben ſie getödtet\\(D-Dl Mus.3388.E.501)}
\instrumentation{2 S, T, B (solo), 2 S, T, B (coro), 2 ob, 2 cor, 2 clno, timp, 2 vl, 2 vla, b, org}
\maketitle


\thispagestyle{empty}

\vspace*{\fill}

\raisebox{-4mm}{\includegraphics{byncsaeu}}\hspace*{1em}Wolfgang Esser-Skala, 2020

© 2020 by Wolfgang Esser-Skala. This edition is licensed under the Creative Commons Attribution-NonCommercial-ShareAlike 4.0 International License. To view a copy of this license, visit \url{http://creativecommons.org/licenses/by-nc-sa/4.0/}. 

Music engraving by LilyPond 2.18.0 (\url{http://www.lilypond.org}).\\
Front matter typeset with Source Sans Pro and Fredericka the Great.

\textit{First version, January 2021}

\vspace*{2cm}

\ifprintreport
\chapter*{Critical Report.}

This edition bases upon a printed score in the Saxon State Library – State and University Library Dresden. The digital version of the manuscript is available at \url{http://digital.slub-dresden.de/id47959385X} (siglum Mus.3388.E.501; see also RISM ID 990069190).

In general, this edition closely follows the printed score. Any changes that were introduced by the editor are indicated by italic type (lyrics, dynamics and directions), parentheses (expressive marks and bass figures) or dashes (slurs and ties). Accidentals are used according to modern conventions. Asterisks denote changes that are clarified in the detailed remarks below.\footnote{Abbreviations: B, bass; b, basses; clno, clarion; cor, horn; Ms, manuscript; ob, oboe; org, organ; r, rest; S, soprano; T,~tenor; timp, timpani; vl, violin; vla, viola.}

\bigskip

\begin{longtable}{lll L{10cm}}
	\toprule
	\itshape Mov. & \itshape Bar & \itshape Staff & \itshape Note \\
	\midrule \endhead
	1 & 31 & vla 1 & 3rd eighth in Ms: es′8 \\
	7 & 80 & vla   & last sixteenth in Ms: es′8 \\
	\bottomrule
\end{longtable}


This edition has been compiled and checked with utmost diligence. Nevertheless, errors and mistakes cannot be totally excluded. Please report any errors and mistakes to \url{wolfgang@esser-skala.at} or create an issue or pull request on the edition’s GitHub page \url{https://github.com/skafdasschaf/wolf-ostercantate}. Your help will be greatly appreciated.

\bigskip
\textit{Salzburg, January 2021\\
Wolfgang Esser-Skala}

\cleardoublepage
\chapter*{Contents.}



\lyrefsection{deslebens}

\begin{description}
	\item[~]
	Des Lebens Fürſten haben ſie getödtet,\\
	den Heiland Iſraels!\\
	Sie nahmen ihn und würgten ihn.\\
	Der Fromme geht dahin, und niemand iſt,\\
	der es zu Herzen nehme.\\
	Der Heilige wird weggerafft,\\
	und niemand achtet drauf.\\
	Aber deine Todten werden leben,\\
	und auferſtehn.\\
	Erwacht, und blüht, ihr Schlafenden\\
	unter der Erde,\\
	ſein Thau iſt Frühlingsthau.
\end{description}

\lyrefsection{allmaechtger}

\begin{description}
	\item[~]
	Allmächtger Schauer dringt durch alle Weſen.\\
	Ringt das Leben und der Tod um ſeinen Fürſten?\\
	Gott Jehova ruft den Sohn\\
	im Schoos der kühlen Nacht.\\
	Vom tiefen Schlaf erwacht,\\
	ſieht auf der Held und blickt empor!\\
	Wer mag ihn halten?\\
	Durch das Thor des Lebens zeucht er!\\
	Helle Schaaren, die in dem Arm der Nacht\\
	gefangen mit ihm waren,\\
	ſie ziehen nach ihm, ihrem Herrn,\\
	wie Sterne nach dem Morgenſtern,\\
	ſie dringen nach dem Licht hervor,\\
	empor, empor!
\end{description}

\lyrefsection{thutauf}

\begin{description}
	\item[~]
	Thut auf die Pforten, die Thore der Welt,\\
	es zeucht der König der Ehren einher!\\
	Wer iſt der König? Es iſt der Held;\\
	ſchrecklich, mächtig im Streit.\\
	Wie kommts, dein Kleid iſt roth von Blut?\\
	Ich trat die Kelter, ich trat ſie allein,\\
	ich ſtritt allein am Tage der Schlacht,\\
	und ward voll Blut!\\
	Thut auf die Pforten, die Thore der Welt,\\
	es zeucht der König der Ehren einher,\\
	und glänzet Heil.
\end{description}

\lyrefsection{jesuschristus}

\begin{description}
	\item[~]
	Jeſus Chriſtus, unſer Heiland,\\
	der den Tod überwand,\\
	iſt auferſtanden,\\
	den Feind hält er gefangen,\\
	Hallelujah!
	
	Tod und Hölle, Leben und Gnad,\\
	all’s in Händen er hat;\\
	er kann erretten\\
	alle, die zu ihm treten,\\
	Hallelujah!
\end{description}

\lyrefsection{wiedie}

\begin{description}
	\item[~]
	Wie die fern abgeſchiedne geliebte Sonne\\
	ſich nach ihres Frühlings Kindern ſehnet,\\
	und wenn in kalter Nacht noch matt ihr Auge thränet,\\
	als Morgenröthe ſchon den düſtern Nebel bricht,\\
	zerreißt der Schleier, und wird Licht,\\
	ſo ſehnet ſich, ſo ſtehet der betrübten Maria Jeſus nah,\\
	und nennt ſie, und iſt da.
	
	Und eilt mit jenem Paar, die nach der Ruhe flehn,\\
	ein Wandrer, mit zu gehn.\\
	Er raubet ſanft ihr Herz, und athmet fremde Glut\\
	in ihren lechzenden, geſunknen, kalten Muth,\\
	enthüllt ſich, und verſchwindet.
	
	Bis er die zehn Geliebten, verlohrenen zuſammen wieder findet,\\
	und Frieden ihnen giebt, und haucht ſie an mit Geiſt,\\
	der von der Balſamkraft des andern Lebens fleußt.
	
	Er ſucht den Irrenden in ſeiner Zweifel Nacht,\\
	der, wie von ſchwerem Traum erwacht,\\
	die Hand ihm legt in ſeine Wunden:\\
	ich habe dich gefunden, mein Herr und Gott!\\
	Du lebeſt, ich bin todt!
	
	Und wandelt in des Morgens Frühe mit ſeinen Kindern:\\
	liebt ihr mich? der mich nicht kannte, Simon, liebſt du mich?\\
	Allwiſſender, o ſiehe mein Herz! Ich liebe dich.
\end{description}

\lyrefsection{siehedas}

\begin{description}
	\item[~]
	Siehe, das ſchöne Morgenroth in dunkler Nacht!\\
	Alſo das Leben durch den Tod erwacht.\\
	Was zagſt du, meine Seele, der kleinen Noth?\\
	Im tiefſten Leiden,\\
	mit Himmelsfreuden\\
	erſcheint dir Gott!
\end{description}

\lyrefsection{derherr}

\begin{description}
	\item[~]
	Der Herr tödtet und machet lebendig,\\
	er führet in die Hölle und führet hinaus.\\\relax
	[1 Samuel 2:6]
\end{description}

\cleardoublepage
\fi

\iftemplate
\includepdf[pages=-]{../out/\lypdfname.pdf}
\fi

\end{document}