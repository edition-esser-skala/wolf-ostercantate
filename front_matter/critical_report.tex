\documentclass[tocstyle=ref-genre]{ees}

\shorttitle{Ostercantate}
\let\oldeesCommentaryIntro\eesCommentaryIntro
\def\eesCommentaryIntro{\oldeesCommentaryIntro\clearpage}

\begin{document}

\eesTitlePage

\eesCriticalReport{
  1  & 31  & vla 1 & 3rd \eighthNote\ in \C1: \flat e′8 \\
  7  & 80  & vla   & last \sixteenthNote\ in \C1: \flat e′8 \\
  11 & 106 & A     & 2nd \quarterNote\ in \C1: b′8.–\sharp a′16 \\
}

\eesToc{
\begin{movement}{deslebens}
  \voice[Coro]
  Des Lebens Fürſten haben ſie getödtet,\\
  den Heiland Iſraels!\\
  Sie nahmen ihn und würgten ihn.\\
  Der Fromme geht dahin, und niemand iſt,\\
  der es zu Herzen nehme.\\
  Der Heilige wird weggerafft,\\
  und niemand achtet drauf.\\
  Aber deine Todten werden leben,\\
  und auferſtehn.\\
  Erwacht, und blüht, ihr Schlafenden\\
  unter der Erde,\\
  ſein Thau iſt Frühlingsthau.
\end{movement}

\begin{movement}{allmaechtger}
  \voice[Soprano {[I]}]
  Allmächtger Schauer dringt durch alle Weſen.\\
  Ringt das Leben und der Tod um ſeinen Fürſten?\\
  Gott Jehova ruft den Sohn\\
  im Schoos der kühlen Nacht.\\
  Vom tiefen Schlaf erwacht,\\
  ſieht auf der Held und blickt empor!\\
  Wer mag ihn halten?\\
  Durch das Thor des Lebens zeucht er!\\
  Helle Schaaren, die in dem Arm der Nacht\\
  gefangen mit ihm waren,\\
  ſie ziehen nach ihm, ihrem Herrn,\\
  wie Sterne nach dem Morgenſtern,\\
  ſie dringen nach dem Licht hervor,\\
  empor, empor!
\end{movement}

\begin{movement}{thutauf}
  \voice[Coro]
  Thut auf die Pforten, die Thore der Welt,\\
  es zeucht der König der Ehren einher!\\
  Wer iſt der König? Es iſt der Held;\\
  ſchrecklich, mächtig im Streit.\\
  Wie kommts, dein Kleid iſt roth von Blut?\\
  Ich trat die Kelter, ich trat ſie allein,\\
  ich ſtritt allein am Tage der Schlacht,\\
  und ward voll Blut!\\
  Thut auf die Pforten, die Thore der Welt,\\
  es zeucht der König der Ehren einher,\\
  und glänzet Heil.
\end{movement}

\begin{movement}{jesuschristus}
  \voice[Coro]
  Jeſus Chriſtus, unſer Heiland,\\
  der den Tod überwand,\\
  iſt auferſtanden,\\
  den Feind hält er gefangen,\\
  Hallelujah!

  Tod und Hölle, Leben und Gnad,\\
  all’s in Händen er hat;\\
  er kann erretten\\
  alle, die zu ihm treten,\\
  Hallelujah!
\end{movement}

\begin{movement}{wiedie}
  \voice[Tenore I]
  Wie die fern abgeſchiedne geliebte Sonne\\
  ſich nach ihres Frühlings Kindern ſehnet,\\
  und wenn in kalter Nacht noch matt ihr Auge thränet,\\
  als Morgenröthe ſchon den düſtern Nebel bricht,\\
  zerreißt der Schleier, und wird Licht,\\
  ſo ſehnet ſich, ſo ſtehet der betrübten Maria Jeſus nah,\\
  und nennt ſie, und iſt da.

  \voice[Tenore II]
  Und eilt mit jenem Paar, die nach der Ruhe flehn,\\
  ein Wandrer, mit zu gehn.\\
  Er raubet ſanft ihr Herz, und athmet fremde Glut\\
  in ihren lechzenden, geſunknen, kalten Muth,\\
  enthüllt ſich, und verſchwindet.

  \voice[Tenore I]
  Bis er die zehn Geliebten, verlohrenen zuſammen wieder findet,\\
  und Frieden ihnen giebt, und haucht ſie an mit Geiſt,\\
  der von der Balſamkraft des andern Lebens fleußt.

  \voice[Tenore II]
  Er ſucht den Irrenden in ſeiner Zweifel Nacht,\\
  der, wie von ſchwerem Traum erwacht,\\
  die Hand ihm legt in ſeine Wunden:\\
  ich habe dich gefunden, mein Herr und Gott!\\
  Du lebeſt, ich bin todt!

  \voice[Tenore I]
  Und wandelt in des Morgens Frühe mit ſeinen Kindern:\\
  liebt ihr mich? der mich nicht kannte, Simon, liebſt du mich?\\
  Allwiſſender, o ſiehe mein Herz! Ich liebe dich.
\end{movement}

\begin{movement}{siehedas}
  \voice[Soprano {[I]}]
  Siehe, das ſchöne Morgenroth in dunkler Nacht!\\
  Alſo das Leben durch den Tod erwacht.\\
  Was zagſt du, meine Seele, der kleinen Noth?\\
  Im tiefſten Leiden,\\
  mit Himmelsfreuden\\
  erſcheint dir Gott!
\end{movement}

\begin{movement}{derherr}
  \voice[Coro]
  Der Herr tödtet und machet lebendig,\\
  er führet in die Hölle und führet hinaus.\\
  (\bibleverse{ISam}(2:6))
\end{movement}

\begin{movement}{nahist}
  \voice[Coro]
  Nah iſt meines Helfers Rechte,\\
  ſieht ſie gleich mein Auge nicht,\\
  weiter hin, im Thal der Nächte,\\
  iſt mein Retter und ſein Licht.\\
  Da, da wird mir Gott begegnen,\\
  da wird mich ſein Antlitz ſegnen,\\
  in der trübſten Stunden Graun,\\
  will ich hoffend nach ihm ſchaun.
\end{movement}

\begin{movement}{oauf}
  \voice[Soprano {[I]}]
  O Auferſtandener, wo ſchwebeſt du, ungeſehn?\\
  In welchem Reiche lebeſt, ein König, Du?\\
  Der Retter der Natur!\\
  Die erſte ſchöne, neuerwachte Blume, auf Gottes Flur!\\
  Und trankſt der Auferſtehung Kraft\\
  für deinen Kelch der Leiden,\\
  einathmend Himmelsfreuden,\\
  verbreitend überall des ewgen Lebens Saft;\\
  ich ſehe Dich.\\
  Dein ſchönes Kleid iſt Morgenroth\\
  in aller Menſchen Blicken,\\
  die Hofnung der Unſterblichkeit;\\
  dein Leid die heilige, verborgne Chriſtenheit;\\
  dein Angeſicht Entzücken.\\
  Ich ſeh! Auf deinem Grabe blüht\\
  des Lebens hoher Baum!\\
  an dem in weitem Raum\\
  die Schöpfung ſich aus Nacht und Moder zieht,\\
  und ewig wächſt und ewig blüht.\\
  Was tönet aus den Grüften\\
  dort für Geſang hervor?\\
  Er ſteiget zu den Lüften;\\
  das Feld der Todten wird der Auferſtehung Chor.
\end{movement}

\begin{movement}{jesusmein}
  \voice[Coro]
  Jeſus mein Erlöſer lebt,\\
  ich werd auch das Leben ſchauen;\\
  ſchweben, wo mein Heiland ſchwebt,\\
  auf des ſchönen Himmels Auen.\\
  Da wird Schwachheit und Verdruß\\
  liegen unter meinem Fuß.
\end{movement}

\begin{movement}{hallelujah}
  \voice[Coro I, II]
  Hallelujah!\\
  Der Tod iſt verſchlungen,\\
  verſchlungen in Siegsgeſang.\\
  Tod, wo iſt dein Pfeil?\\
  Grab, wo iſt dein Sieg?\\
  Gelobt ſey Gott,\\
  der uns den Sieg gegeben\\
  durch Chriſtum, unſern Herrn.
\end{movement}
}

\eesScore

\end{document}
